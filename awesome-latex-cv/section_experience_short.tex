% Awesome CV LaTeX Template
%
% This template has been downloaded from:
% https://github.com/huajh/huajh-awesome-latex-cv
%
% Author:
% Junhao Hua
%Section: Work Experience at the top
\sectionTitle{Projects \& Experiences}{\faCode}
 \setlength{\parskip}{10pt plus 1pt minus 1pt}

\begin{experiences}
    \small
    \experience
        {\textbf{Present}}   % End date
        {\textbf{Research \& Development Engineer II}}  % Position
        {\textbf{Commscope}}  % Company
        {\textbf{Bengaluru}}  % Location
        {   % Start date
            \textbf{Jan 2022}  % Assuming this is the start date
        }
        {   % Project Experience
            \begin{itemize}
                \item \textbf{Yocto BSP and Cloud RAN Development:} Owned Yocto BSP and image customization for Cloud RAN appliances; streamlined layer hygiene and CI, reducing image build time by \textbf{22\%} and field rollouts by \textbf{~1 day/release}.
                \item \textbf{PS-Side Drivers \& Device Control:} Implemented PS-side drivers and user-space control paths (I2C/SPI, DMA, IRQ) for radio subsystems; cut bring-up defects by \textbf{30\%}.
                \item \textbf{MIPI CSI-2 Camera Pipeline:} Developed MIPI CSI-2 camera capture path on Embedded Linux (V4L2, media graph, sensor init, lane config); stabilized streaming with \textbf{0 dropped frames} in 30-min soak.
                \item \textbf{RAN Microservices \& Containerization:} Containerized RAN managers (DM/AM/PFM) with \textbf{gRPC} interfaces; deployed on \textbf{Kubernetes}; integrated \textbf{Kafka} log streaming and probes for SLOs.
                \item \textbf{Test Automation \& CI/CD:} Established Robot Framework + Jenkins test automation across multi-node rigs, increasing regression coverage from \textbf{~45\%} to \textbf{>80\%}.
            \end{itemize}
        }
        {}
    \experience
        {\textbf{Jan 2022}}   % End date
        {\textbf{Software Engineer}}  % Position
        {\textbf{Capgemini}}  % Company
        {\textbf{Gurgaon}}  % Location
        {   % Start date
            \textbf{Nov 2018}  % Assuming this is the start date
        }
        {   % Project Experience
            \begin{itemize}
                \item \textbf{Automotive Embedded Software (AUTOSAR):} Developed automotive embedded software using \textbf{AUTOSAR} architecture for instrument clusters; implemented telltale and door warning applications with Embedded C and Vector tools (CANalyzer, CANoe, DaVinci).
                \item \textbf{Comprehensive Testing Strategy:} Executed comprehensive testing strategy spanning Unit, Functional, Integration, and System levels; enhanced software reliability and compliance with automotive standards.
                \item \textbf{Smart-Grid OTA Applications:} Designed Smart-Grid Network Interface Card applications featuring \textbf{Over-The-Air (OTA)} updates and commissioning processes; improved field deployment efficiency by \textbf{25\%}.
                \item \textbf{Configuration Management:} Maintained code quality and documentation using JIRA, DOORS, and Enterprise Architect; established coding standards that reduced integration defects by \textbf{40\%}.
            \end{itemize}    
        }
        {}       
        
\experience
  {Aug 2018}   % End date
  {IoT Support Engineer}  % Position
  {Buffalogrid Project Pvt Ltd}  % Company
  {Delhi}  % Location
  {Jan 2018}  % Start date
  {   % Project Experience
      \begin{itemize}
          \item \textbf{Solar-Powered Hub Firmware:} Developed firmware for distributed solar-powered mobile charging hubs; implemented \textbf{FOTA updates} and Battery Management System (BMS) software for field-deployed units.
          \item \textbf{Test Automation:} Created test automation scripts reducing manual QA effort by \textbf{40\%}; enhanced product testing reliability and deployment validation.
      \end{itemize}    
  }
  {}  % This empty bracket can be removed if there is no content, or it can be used for technologies used.

  \experience
  {Dec 2017}   % End date
  {Embedded Software Engineer}  % Position
  {Eigen Technologies Pvt Ltd}  % Company
  {Delhi}  % Location
  {Dec 2015}  % Start date
  {   % Project Experience
      \begin{itemize}
          \item \textbf{WSN \& BLE Product Development:} Led electronic design and firmware development for WSN-based smart streetlight and BLE-based home automation products; delivered end-to-end IoT solutions.
          \item \textbf{Multi-Protocol Firmware:} Implemented multi-protocol firmware supporting 802.15.4 protocols including Zigbee; conducted comprehensive QA testing ensuring compatibility across devices.
          \item \textbf{AWS IoT Dashboard:} Architected and deployed IoT dashboard on AWS with real-time data visualization; integrated with 4G module gateway using AT commands for remote monitoring.
      \end{itemize}    
  }
  {}  % Technologies used (if any).

    
\end{experiences}

% Selected Low-Level Systems Projects
\sectionTitle{Selected Low-Level Systems Projects}{\faListUl}
\textbf{Yocto BSP for Custom Zynq Board} — Distro layers, U-Boot patches, kernel config, image recipes; reduced boot time via systemd unit profiling and init sequence trimming. 
\newline
\textbf{MIPI CSI-2 Camera Driver \& Pipeline} — Sensor init (I2C), CSI lane timing, V4L2 sub-dev, media controller graph; validated with long-run soak, artifact-free frames.
\newline
\textbf{PS-Side DMA/Interrupt Path} — Engineered robust DMA ring buffers and IRQ service; back-pressure handling to maintain deterministic throughput.
\newline
\textbf{RAN Microservices on K8s} — gRPC services (DM/AM/PFM), Kafka log bus, readiness/liveness probes; Helm deploys and blue/green updates.
\newline
\textbf{HW Bring-Up \& Debug} — Pinmux, clock tree, PMIC init; boundary scan + JTAG; logic analyzer traces for ISR latency verification.